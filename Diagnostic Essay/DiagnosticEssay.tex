\documentclass[11pt]{article}

\usepackage{geometry}
\geometry{
	letterpaper,
	left=1.1in,
	right=1.1in,
	top=1.1in,
	bottom=1.1in,
	}

\usepackage{setspace}
\doublespacing

\usepackage{textcomp}

\usepackage{gensymb}

\setlength{\parindent}{15pt}

\begin{document}

\title{
	Reducing Resource Consumption and Carbon Footprints\\
	Engineering 100 Diagnostic Essay
	}

\author{
	Rishabh Shah\\
	rishabas\\
	UM ID: 4655 4192}

\date{January 10, 2017}

\maketitle
\thispagestyle{empty}

\newpage
\clearpage
\pagenumbering{arabic}


``Does anyone know the significance of the number 350?'' Jay asked in front of the whole school late sophomore year. No one really knew what was going on, so he stood there at the front of the auditorium until the silence was unbearable. ``It's the number of CO\textsubscript{2} particles per million that we should be at to sustain life. We're at 400 and it keeps rising.'' His announcement was to hype up a screening of `Do The Math', a documentary about the changing climate and the rising amount of CO\textsubscript{2} in our atmosphere. I never did end up going to watch that documentary, but it did start turning some gears in my brain as to how I could reduce my resource consumption and my carbon footprint.

At home, reducing my carbon footprint and resource consumption was difficult to think about at first. A lot things my family was doing to save money also happened to directly correlate to a lower carbon footprint. During the winter, the house was heated with natural gas, which is more efficient than home heating oil or coal, and was heated to 66\degree \ F. Along with the heating, the hot water for the house has always been heated by a gas water heater. When the old one broke, it was replaced with a more efficient on-demand water heater. My entire family is vegetarian, thus cutting out the meat industry from our resource consumption. Likewise, during the summer, we had a vegetable garden which was irrigated by rainwater. Although it was not large enough to fulfill a majority of our produce needs, it did help reduce our resource use. In addition, we did our grocery shopping at a store that was committed to stocking as much local produce as it could. Both my younger brother and I have never owned or been given a car, under the pretense that our transportation needs could be satisfied by public transport or carpooling.

However, there was and still is a great room for improvement. After doing some research into reducing resource consumption, one of the easiest things I realized my family could do is to replace all of our incandescent and CFL light bulbs with LED ones. The State of Massachusetts was even running a program through National Grid, the electric provider that services our town, which would allow us to get our entire house outfitted with LED light bulbs, ``free'' of charge. According to our electric bill, the usage for the next month dropped drastically, again reducing our resource consumption and consequently our carbon footprint. While continuing to research ways my family could ``go greener'', there were quite a few options which in theory would lower our carbon footprint, but for monetary or time constraints were not feasible. These included purchasing more efficient vehicles for my parents, such as diesel sedans; installing solar panels on our roof; resealing all the windows in the house; in-home liquid and solid waste recycling; and composting the majority of the waste created. Unfortunately, none of these options have been acted on yet, however, my parents are planning on installing solar panels as the State is offering substantial rebates for households who make the transition.

After moving into University Housing in August, I again found myself surprised. The University was doing a lot on its part in trying to keep resource consumption down, but at the same time, there is still quite a large amount of waste. To begin, Bursley and Markley Dining are the only two dining halls that I have come across that have compost bins for food waste. Although Markley Dining has the option to compost, I have observed that very few students use the compost bin, which could be due to a variety of factors (the bin itself may not be visible enough or students aren't being educated in its use for example). In my opinion, the Bursley Dining Hall has the optimal setup for increasing compost usage instead of simply throwing the food in the trash---it essentially forces the majority of students to compost their food waste rather than just throw it out. If this system was implemented in more dining halls, I am certain that less landfill space would be used. Further, the University could install a nuclear power station instead of the current oil and natural gas burning Central Power Plant. A nuclear power station would most likely have a larger footprint than the current power station, which may increase immediate resource consumption, but once the plant is operational, the plant itself would output a total of zero greenhouse gases. Of course, the extraction of Uranium from the Earth has a quite a large carbon footprint, but when compared to oil, it is overall significantly less. However, the only obstacles to a nuclear power station are the large upfront capital needed to build one, in addition to convincing the surrounding public to look at the numbers rather than the urban legends. In addition, the owner must overcome the strict government regulations against nuclear power stations, most likely heavily `influenced' by the oil and natural gas industry.

Although ``going green'' may sound like something easily accomplished, there is always another way to lower carbon emissions and reduce resource consumption. Whether that is the first step or the thousandth step towards ``going green'', we must continue to strive to take the next step. Even though the ``green'' movement is highly politicized, it is important to keep the movement's goals in mind when discussing the extremely polarized topic. In my understanding, the movement is about inflicting the least damage to all life, including other humans. The only way to achieve this at the end of the day is if every individual continually works towards improving themselves.

\end{document}